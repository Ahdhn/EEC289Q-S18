\documentclass[12pt] {article}
\usepackage{times}
\usepackage[margin=1in,bottom=1in,top=0.6in]{geometry}

\usepackage{hhline}
\usepackage{subfig}
\usepackage{amsmath}
\usepackage{amsfonts}
\usepackage[inline,shortlabels]{enumitem}%enumerate with letters
\usepackage{mathrsfs} 
\usepackage[square,numbers]{natbib}
\usepackage{graphicx}
\bibliographystyle{unsrtnat}
\usepackage{float}
\usepackage[framed,numbered,autolinebreaks,useliterate]{../mcode}

\begin{document}

\title{EEC 289Q – Data Analytics for Computer Engineers \\ Homework 3}
\author{Ahmed Mahmoud}
\date{May, 7th 2018} 

\maketitle




%============Table========
%\begin{table}[tbh]
% \centering    
%\begin{tabular}{ |p{4cm}|| p{2cm}|p{2cm}|p{2cm}|p{2cm}|}
% \hline
% & Processor 1 &  Processor 2  & Processor 3 & Processor 4\\ \hhline{|=|=|=|=|=|}
% \hline
% Performance          &$1.08$        &$1.425$       &\textbf{1.52}  &   \\
% \hline
%\end{tabular} 
%\caption{Metric table for the four processors}
%   \label{tab:metric}
%\end{table} 
%============Figure========
%\begin{figure}[!tbh]
%\centering        
%   \subfloat {\includegraphics[width=0.65\textwidth]{fig2_4.png}}
%   \caption{ }
%   \label{fig:fig}
%\end{figure}


\section*{Softmax Regression:} 
The following code shows the implementation of the softmax regression function
\begin{lstlisting}
function [f,g] = softmax_regression(theta, X,y) 
	m=size(X,2);
	n=size(X,1);
	theta=reshape(theta, n, []);
	num_classes=size(theta,2)+1;
	f = 0;
	g = zeros(size(theta));
%%% YOUR CODE HERE %%%
	theta_x = exp(theta'*X);
	sum_col = sum(theta_x);
	for I=1:m
    	theta_x(:,I) = theta_x(:,I)/sum_col(I);
	end
	for I=1:m
    	 if y(I) <num_classes
        	 f = f - log(theta_x(y(I),I));
	     end
	end
	%expand y to matrix to allow matrix multiply to obtain the gardient 
	y_mat = full(sparse(y,1:m,1));
	g = -X * (y_mat(1:num_classes-1,:) - theta_x)';
	g=g(:); % make gradient a vector for minFunc
\end{lstlisting}

Using this code, we were able to achieve training accuracy of $87.2\%$ and 
test accuracy of $87.6\%$ while the optimization took $4.524218 $ seconds. We used the gradient checker on our implementation and the average absolute error was $0.0330$ (of 10 tests).

\newpage

\section*{Supervised Neural Networks:}
The following code implements the cost function, forward propagation, and compute the gradients for multiple hidden layers neural network 


\begin{lstlisting}
function [ cost, grad, pred_prob] = supervised_dnn_cost( theta, ei, data, labels, pred_only)
	%SPNETCOSTSLAVE Slave cost function for simple phone net
	%   Does all the work of cost / gradient computation
	%   Returns cost broken into cross-entropy, weight norm, and prox reg
	%        components (ceCost, wCost, pCost)

	%% default values
	po = false;
	if exist('pred_only','var')
	  po = pred_only;
	end

	%% reshape into network
	stack = params2stack(theta, ei);
	numHidden = numel(ei.layer_sizes) - 1;
	hAct = cell(numHidden+1, 1);
	gradStack = cell(numHidden+1, 1);
	%% forward prop
	%%% YOUR CODE HERE %%%
	for J=1:numHidden
    	if J==1
        	%input
	        z = stack{J}.W*data; 
	    else 
    	    %activiation
	        z = stack{J}.W*hAct{J-1};
    	end
	    z = z + stack{J}.b;
	    hAct{J}=sigmoid(z);
	end
	z = stack{numHidden+1}.W*hAct{numHidden};
	z = z + stack{numHidden+1}.b;
	E = exp(z);
	pred_prob = E./sum(E,1);
	hAct{numHidden+1} = pred_prob;

	%% return here if only predictions desired.
	if po
	  cost = -1; ceCost = -1; wCost = -1; numCorrect = -1;
	  grad = [];  
	  return;
	end
	%% compute cost
	%%% YOUR CODE HERE %%%
	c = log(pred_prob);
	ind =sub2ind(size(c), labels', 1:size(c,2));
	values = c(ind);
	ceCost = -sum(values);
	%% compute gradients using backpropagation
	%%% YOUR CODE HERE %%%
	d = zeros(size(pred_prob));
	d(ind)=1;
	error = (pred_prob-d);
	for l =numHidden+1:-1:1
	    gradStack{l}.b = sum(error,2);
    	if l ==1
        	gradStack{l}.W = error*data';
	        break;
	    else
    	    gradStack{l}.W = error*hAct{l-1}';
	    end
	    error = (stack{l}.W)'*error.*hAct{l-1}.*(1-hAct{l-1});    
	end
	%% compute weight penalty cost and gradient for non-bias terms
	%%% YOUR CODE HERE %%%
	wCost = 0;
	for l = 1:numHidden+1
    	wCost = wCost + 0.5*ei.lambda*sum(stack{l}.W(:).^2);
	end
	cost = ceCost + wCost;
	for l=numHidden:-1:1
    	gradStack{l}.w = gradStack{l}.W + ei.lambda*stack{l}.W;
	end
	%% reshape gradients into vector
	[grad] = stack2params(gradStack);
end
\end{lstlisting}

Using this code, we were able to get train accuracy of $100\%$ and test accuracy of $0.9712\%$ with weight decay value of 0.0. We can have a better test accuracy by changing the weight decay value to 0.25 which gave test accuracy of $0.973800\%$ while the train accuracy lowered to $0.997717\%$. We think this is due to over-fitting with weight decay of 0. We used different values of weight decay with two and four hidden layers but not has shown superior performance. Also, changing/reduces the layer size always produced lower accuracy. 

\newpage
\section*{AlexNet:}
\paragraph{First Layer:} The input to AlexNet is a $[227 \times 227 \times 3]$ image (weight $W$). The first convolutional layer has receptive field ($F$) of 11, stride ($S$) of 4 and no zero-padding ($P$) with 96 kernels ($K$) (or  convolutional layer depth of 96). Thus, the output is 	
$$
\frac{W-F+2P}{S} + 1 = \frac{227 - 11 + 2*0}{4} + 1 = 55
$$
The number of neurons in this layer is $55*55*96=290,400$. Each of the $55\times 55$ slice uses a unique $11*11*3=363$ weights and $1$ bias. Thus, the total number of parameters in the first layer is $96*363 + 96 = 34,944$ parameters. 

The operations for this layers are as follows: in order to obtain the one value in the $55 \times 55 \times 96$ output, the filter application will require $11\times 11$multiply operations and 11 add operations to do the dot product. This will be done 3 times for the three color channels in the input and add up all the result together (3 multiply) in addition to 1 add for the bias. Thus, the number of operations for one entry is $((11*11 + 11)*3)+1=397$ operations. For the whole layer, the total number of operations is $(55*55*96)*397= 115,288,800$. 

\emph{Note:} If we consider multiply and add as one operation (taking one cycle), then the number of operations per one entry is reduced to be $11*11*3+1=364$. This will make the total number of computation in this layer to be $(55*55*96)*364=105,705,600$.

\paragraph{Second Layer:} After the max pooling in the first layer, the input to the second layer becomes $[27 \times 27 \times 96]$. Second convolutional layer has receptive field ($F$) of 5, stride ($s$) of 1 and zero-padding ($P$) of 2 with 256 kernels ($K$) . Thus, the output is 
$$
\frac{W-F+2P}{S} + 1 = \frac{27-5+2*2}{1} + 1 = 27
$$
The number of neurons in this layers is $27*27*256=186,624$. Each of the $27 \times 27$ slice uses a unique set of weight of size $5*5*96=2,400$ weights and $1$ bias. Thus, the total number of parameters in the second layer is $256*2,400 + 256= 614,656$ parameters.

Following the same computation we have done for the first layer, the total number of computation done in the second layer is $(27*27*256)* \left( \left( (5*5+5)*96 \right) +1 \right) = 537,663,744$.

\paragraph{Third Layer:} After max pooling, the input to the third layer is $[13 \times 13 \times 256]$. The third convolutional layer has receptive field of ($F$) 3, stride ($S$) of 1 and zero-padding ($P$) of 1 with 384 kernels ($K$). Thus, the output is
$$
\frac{W-F+2P}{S} + 1 = \frac{13-3+2*1}{1} + 1 = 13
$$

The  number of neurons in this layer is $13*13*384 = 64,896$. Each of the $13 \times 13$ slice uses a unique set of weight of size $3*3*256 =2,304$ weights and 1 bias. Thus, the total number of parameters in the third layer is $2,304*384 + 384 = 885,120$ parameters. 

The total computation in the third layer is $(13*13*384)*\left(  \left( (3*3+3)*256 \right)+1 \right) = 199,425,408$. 

\paragraph{Fourth Layer:} The output of third layer goes straight to fourth layer. The fourth convolutional layer has receptive field of ($F$) 3, stride ($S$) of 1 and zero-padding ($P$) of 1 with 384 kernels ($K$). The output size is (similar to third layer) 13.

The number of neurons is $13*13*384 = 64,896$. Each of this $13 \times 13$ slice uses a unique set of weights of size $3*3*384=3,456$ weights and 1 bias. Thus, the total number of parameters in the fourth layer is $3,456*384 + 384 =1,327,488$.

The total computation in the forth layer is 

$(13*13*384)*\left(  \left( (3*3+3)*384 \right)+1 \right) = 299,105,664$. 

\paragraph{Fifth Layer:} The output of fourth layers goes straight to the fifth layer. The fifth convolutional layer has receptive field of ($F$) 3, stride ($S$) of 1 and zero-padding ($P$) of 1 with 256 kernels ($K$). The output size is 13.


The number of neurons is $13*13*256 =43,264$. Each of this $13\times 13$ slice uses a unique set of weight of size $3*3*384=3,456$ weights and 1 bias. Thus, the total number of parameters in the fifth layer is $3,456*256+256=884,992$. 

The total computation in the fifth layer is $(13*13*256)*\left(  \left( (3*3+3)*384 \right)+1 \right) = 199,403,776$. 

\paragraph{Sixth Layer:} After max pooling in the fifth layer, the input to the sixth fully connect (FC) layer is $[6\times 6\times 256]$. The total number of neurons of the sixth layer is $4096$. Since it is fully connected, the number of parameters is $6*6*256*4096 = 37,748,736$. The number of operations for one neuron is $6*6*256$ multiply followed by $6*6*256$ accumulate/sum. Thus, the total number of operations are $(6*6*256)*2*4096=75,497,472$.



\paragraph{Seventh Layer:} The output of the sixth layer goes to the seventh layer i.e., the input to the seventh layer is $[4096]$. The total number of neurons of the seventh layer is $4096$. Since it is fully connected, the number of parameters is $4096*4096 = 16,777,216$. The total number of operations are $4096*2*4096= 33,554,432$.




\paragraph{Eighth Layer:} The output of the seventh layer goes to the eighth layer i.e., the input to the eighth layer is $[4096]$. The total number of neurons of the eighth layer is $1000$. Since it is fully connected, the number of parameters is $4096*1000 = 4,096,000$. The total number of operations are $4096*2*1000= 8,192,000$.
\\


Table~\ref{tab:metric} shows the number of neurons, parameters, and operations for all layers. We notice that $~94\%$ of the weights are in the fully connected layers $~89\%$ of the computation (and $~98\%$ of the neurons) are withing the convolutional layers.



\begin{table}[tbh]
 \centering    
\begin{tabular}{ |p{1.5cm}||p{4.2cm}|p{4.2cm}|p{4.2cm}|}
 \hline
  Layer & Neurons & Parameters  & Operations \\ \hhline{|=|=|=|=|}
 \hline
  $\#1$   &  290,400(44.0481\%)&  34,944     (0.0560\%)&  115,288,800(7.8527\%) \\
   \hline                                                            
  $\#2$   &  186,624(28.3071\%)&  614,656    (0.9855\%)&  537,663,744(36.6223\%)\\
   \hline                                                            
  $\#3$   &  64,896 (9.8435\%)&  885,120    (1.4191 \%)&  199,425,408(13.5836\%)\\
   \hline                                                            
  $\#4$   &  64,896 (9.8435\%)&  1,327,488  (2.1284 \%)&  299,105,664(20.3732\%)\\
   \hline                                                            
  $\#5$   &  43,264 (6.5623\%) &  884,992   (1.41895\%)&  199,403,776(13.5821\%) \\
   \hline                                                            
  $\#6$   &  4096   (0.6212\%)&  37,748,736 (60.5246\%)&  75,497,472 (5.1424\%) \\
   \hline                                                            
  $\#7$   &  4096   (0.6212\%)&  16,777,216 (26.8998\%)&  33,554,432 (2.2855\%)  \\
   \hline                                                             
  $\#8$   &  1000   (0.1516\%)&  4,096,000  (6.5673 \%)&  8,192,000  (0.5579\%)  \\ \hhline{|=|=|=|=|}
  Total   &  659,272 & 62,369,152  &  1,468,131,296 \\ 
 \hline
\end{tabular} 
\caption{AlexNet absolute number of neurons, parameters, and operations for the convolutional and fully connected layers (and percentage).}
   \label{tab:metric}
\end{table} 


\end{document} 














































