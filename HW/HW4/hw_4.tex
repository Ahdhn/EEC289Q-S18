\documentclass[12pt] {article}
\usepackage{times}
\usepackage[margin=1in,bottom=1in,top=0.6in]{geometry}

\usepackage{hhline}
\usepackage{subfig}
\usepackage{amsmath}
\usepackage{amsfonts}
\usepackage[inline,shortlabels]{enumitem}%enumerate with letters
\usepackage{mathrsfs} 
\usepackage[square,numbers]{natbib}
\usepackage{graphicx}
\bibliographystyle{unsrtnat}
\usepackage{float}
\usepackage[framed,numbered,autolinebreaks,useliterate]{../mcode}

\begin{document}

\title{EEC 289Q – Data Analytics for Computer Engineers \\ Homework 4}
\author{Ahmed Mahmoud}
\date{May, 25th 2018} 

\maketitle




%============Table========
%\begin{table}[tbh]
% \centering    
%\begin{tabular}{ |p{4cm}|| p{2cm}|p{2cm}|p{2cm}|p{2cm}|}
% \hline
% & Processor 1 &  Processor 2  & Processor 3 & Processor 4\\ \hhline{|=|=|=|=|=|}
% \hline
% Performance          &$1.08$        &$1.425$       &\textbf{1.52}  &   \\
% \hline
%\end{tabular} 
%\caption{Metric table for the four processors}
%   \label{tab:metric}
%\end{table} 
%============Figure========
%\begin{figure}[!tbh]
%\centering        
%   \subfloat {\includegraphics[width=0.65\textwidth]{fig2_4.png}}
%   \caption{ }
%   \label{fig:fig}
%\end{figure}


\section*{Convolution and Pooling:} 
The following code shows the implementation convolution operation
\begin{lstlisting}
function convolvedFeatures = cnnConvolve(filterDim, numFilters, images, W, b)

	numImages = size(images, 3);
	imageDim = size(images, 1);
	convDim = imageDim - filterDim + 1;
	convolvedFeatures = zeros(convDim, convDim, numFilters, numImages);

	for imageNum = 1:numImages
	  for filterNum = 1:numFilters    	
	    convolvedImage = zeros(convDim, convDim);

    	%%% YOUR CODE HERE %%%
	    filter = W(:,:,filterNum);
	    filter = rot90(squeeze(filter),2);      
	    im = squeeze(images(:, :, imageNum));

    	%%% YOUR CODE HERE %%%
	    convolvedImage = conv2(im, filter,'valid');

	    %%% YOUR CODE HERE %%%
    	convolvedImage = convolvedImage + b(filterNum);
	    convolvedImage  = sigmoid(convolvedImage);    
    	convolvedFeatures(:, :, filterNum, imageNum) = convolvedImage;
    	
	  end	
	end
end
\end{lstlisting}
\newpage
The following code shows the implementation average pooling operation
\begin{lstlisting}
function pooledFeatures = cnnPool(poolDim, convolvedFeatures)

	numImages = size(convolvedFeatures, 4);
	numFilters = size(convolvedFeatures, 3);
	convolvedDim = size(convolvedFeatures, 1);

	pooledFeatures = zeros(convolvedDim / poolDim, ...
    	    convolvedDim / poolDim, numFilters, numImages);

	%%% YOUR CODE HERE %%%
	pool_window = ones(poolDim)/poolDim^2;
	out_size = 1:poolDim:convolvedDim;
	for imageNum = 1:numImages
    	for filterNum = 1:numFilters
			conv_out = conv2(convolvedFeatures(:,:,filterNum,imageNum),...
                pool_window,'valid');
	        pooledFeatures(:,:,filterNum,imageNum) = conv_out(out_size,out_size);
	    end
	end
end
\end{lstlisting}

These two pieces of codes have passed the tests in $\mathrm{cnnExercise.m}$ successfully.
\newpage
\section*{Convolutional Neural Network:} 
Our implementation for the CNN is shown in the following four codes/functions. Our implementation passed the gradient check with a difference between numerical gradient and ours of $ 2.3679 \times 10^{-10}$ which is less than the tolerance (i.e., $ 10^{-9}$). After training, the accuracy of our implementation after three epochs was $97.26\%$ and the training took in average 4 minutes. 



\begin{lstlisting}
function [cost, grad, preds] = cnnCost(theta,images,labels,numClasses,...
                                filterDim,numFilters,poolDim,pred)
	if ~exist('pred','var')
    	pred = false;
	end;


	imageDim = size(images,1); % height/width of image
	numImages = size(images,3); % number of images

	%% Reshape parameters and setup gradient matrices
	[Wc, Wd, bc, bd] = cnnParamsToStack(theta,imageDim,filterDim,numFilters,...
    	                    poolDim,numClasses);

	% Same sizes as Wc,Wd,bc,bd. Used to hold gradient w.r.t above params.
	Wc_grad = zeros(size(Wc));
	Wd_grad = zeros(size(Wd));
	bc_grad = zeros(size(bc));
	bd_grad = zeros(size(bd));

	convDim = imageDim-filterDim+1; % dimension of convolved output
	outputDim = (convDim)/poolDim; % dimension of subsampled output

	% convDim x convDim x numFilters x numImages tensor for storing activations
	activations = zeros(convDim,convDim,numFilters,numImages);

	% outputDim x outputDim x numFilters x numImages tensor for storing
	% subsampled activations
	activationsPooled = zeros(outputDim,outputDim,numFilters,numImages);

	%%% YOUR CODE HERE %%%
	activations = cnnConvolve(filterDim, numFilters, images, Wc, bc);
	activationsPooled = cnnPool(poolDim,activations);

	% Reshape activations into 2-d matrix, hiddenSize x numImages,
	% for Softmax layer
	activationsPooled = reshape(activationsPooled,[],numImages);

	probs = zeros(numClasses,numImages);

	%%% YOUR CODE HERE %%%
	soft = Wd*activationsPooled + repmat(bd,1,numImages);
	soft = exp(soft - max(soft,[],1));
	probs = soft./sum(soft);

	cost = 0; % save objective into cost

	%%% YOUR CODE HERE %%%
	%one-hot encoding 
	hot = full(sparse(labels,1:numImages,1));
	cost = -hot(:)'*log(probs(:))/numImages;
	% Makes predictions given probs and returns without backproagating errors.
	if pred
    	[~,preds] = max(probs,[],1);
	    preds = preds';
	    grad = 0;
	    return;
	end;

	%%% YOUR CODE HERE %%%
	dalta = probs-hot;
	term1 = Wd'*((1/numImages).*dalta);
	err_tmp = reshape(term1,outputDim,outputDim,numFilters,numImages);
	for I=1:numImages 
    	for F=1:numFilters        
	        err(:,:,F,I) = (1/(poolDim^2)).*kron(squeeze(err_tmp(:,:,F,I))...
            			,ones(poolDim,poolDim));        
    	end
	end
	err = activations.*(1.0 -activations).*err;

	%%% YOUR CODE HERE %%%
	for I=1:numImages
    	for F=1:numFilters
			Wc_grad(:,:,F) = Wc_grad(:,:,F)+conv2(images(:,:,I),...
            		rot90(err(:,:,F,I),2),'valid');
	        bc_grad(F) = bc_grad(F) + sum(sum(err(:,:,F,I)));
    	end
	end
	Wd_grad= dalta*(activationsPooled)'/numImages;
	bd_grad = (1/numImages).*dalta*ones(numImages,1);

	%% Unroll gradient into grad vector for minFunc
	grad = [Wc_grad(:) ; Wd_grad(:) ; bc_grad(:) ; bd_grad(:)];
end
\end{lstlisting}

\newpage

\begin{lstlisting}
function [opttheta] = minFuncSGD(funObj,theta,data,labels,...
                        options)
	assert(all(isfield(options,{'epochs','alpha','minibatch'})),...
    	    'Some options not defined');
	if ~isfield(options,'momentum')
    	options.momentum = 0.9;
	end
	epochs = options.epochs;
	alpha = options.alpha;
	minibatch = options.minibatch;
	m = length(labels); % training set size
	% Setup for momentum
	mom = 0.5;
	momIncrease = 20;
	velocity = zeros(size(theta));

	%%=====================================================
	%% SGD loop
	it = 0;
	for e = 1:epochs
    
    	% randomly permute indices of data for quick minibatch sampling
	    rp = randperm(m);
    
    	for s=1:minibatch:(m-minibatch+1)
        	it = it + 1;
	        % increase momentum after momIncrease iterations
    	    if it == momIncrease
        	    mom = options.momentum;
	        end
	        % get next randomly selected minibatch
    	    mb_data = data(:,:,rp(s:s+minibatch-1));
        	mb_labels = labels(rp(s:s+minibatch-1));
	        % evaluate the objective function on the next minibatch
    	    [cost, grad] = funObj(theta,mb_data,mb_labels);
        
	        %%% YOUR CODE HERE %%%
    	    velocity =(mom.*velocity)+(alpha.*grad);
        	theta = theta - velocity;
        
	        fprintf('Epoch %d: Cost on iteration %d is %f\n',e,it,cost);
    	end

	    % aneal learning rate by factor of two after each epoch
    	alpha = alpha/2.0;
	end
	opttheta = theta;
end

\end{lstlisting}


\end{document} 

